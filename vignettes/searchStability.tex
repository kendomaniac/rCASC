\title{Stability Score Algorithm}
\author{
      Luca Alessandr\i'
}
\date{\today}

\documentclass[12pt]{article}
\usepackage{amsmath}

\begin{document}
\setcounter{page}{34}



\paragraph{Stability score Algorithm}
Be $C$ the count matrix with $NXM$ genes and cells.
\[
C = \begin{bmatrix} 
    c_{11} & c_{12} & \dots \\
    \vdots & \ddots & \\
    c_{N1} &        & c_{NM} 
    \end{bmatrix}
\]
Be $Cl_p$ the result vector of the clustering algorithm in the permutation $p$.



 \[
   \left\{
                \begin{array}{ll}
                  p = 0 \ then \ C \ is \ complete\\
                  p \neq 0 \ then \ C \ is \ a \ subset \ of \ C. A \ specific\ percentage\ of\ cell\ is\ removed. 
                \end{array}
              \right.
  \]
  
 Be $R_p$ the relation simmetric matrix $MXM$ generated from $Cl_p$   
\[
R_p = \begin{bmatrix} 
    Rp_{11} & Rp_{12} & \dots \\
    \vdots & \ddots & \\
    Rp_{M1} &        & Rp_{MM} 
    \end{bmatrix}
\]

 \[
   R_p(m_x,m_y)=\left\{
                \begin{array}{ll}
                 1 \ if \ Cl_p(m_x)=Cl_p(m_y)\\
                 0 \ otherwise 
                \end{array}
              \right.
  \]

Be $R_pC = R_0 + R_p$ ,$S$ the threshold of the score percentage, be dim() the function that detect how many elements are in a vector,$v_{p_m}$ each row of $R_pC$, k an interger number and be Length() the function 
 \[
   length(v,k)=\sum\limits_{1}^{dim(v)}\left\{
                \begin{array}{ll}
                 1 \ if \ R_pC_{v} = k\\
                 0 \ otherwise 
                \end{array}
              \right.
  \]
  
  $ scoreT_{p_m}= \frac{ length(v_{p_m},2)}{length(v_{p_m},2)+length(v_{p_m},1)}$ 
$\forall \ m  \land \forall \ p $
 \[
   scoreTT_{p_m}=\left\{
                \begin{array}{ll}
                 1 \ if scoreT_{p_m} \geq S \\
                 0 \ otherwise 
                \end{array}
              \right.
  \]
  
Be $P$ the total number of permutation 

$Score_m= \sum\limits_{p=1}^P \frac{scoreTT_{p_m}}{P} $


\section{Example}
Be $Cl(p,m)= \begin{Bmatrix}
 1 & 1 & 1 & 1 & 1 \\
 2 & 2 & 2 & 2 & 2 \\
 2 & 1 & 1 & 1 & 2 \\
 1 & 1 & 2 & 1 & 1 \\
 2 & 2 & 2 & 2 & 2 
\end{Bmatrix}  $

$\forall \ p  \ R_p $ is calculated, 

$ R_0=\begin{bmatrix}
1 & 0 & 1 & 1 & 0 \\
0 & 1 & 0 & 0 & 1 \\
1 & 0 & 1 & 1 & 0 \\
1 & 0 & 1 & 1 & 0 \\
0 & 1 & 0 & 0 & 1 
\end{bmatrix}  $

$ R_1=\begin{bmatrix}
1 & 0 & 0 & 1 & 0 \\
0 & 1 & 1 & 0 & 1 \\
0 & 1 & 1 & 0 & 1 \\
1 & 0 & 0 & 1 & 0 \\
0 & 1 & 1 & 0 & 1 
\end{bmatrix}  $
\dots $R_p$
\\
$ \forall p > 0 \ R_pC \ is \ calculated$
$ R_1C=\begin{bmatrix}
2 & 0 & 1 & 2 & 0 \\
0 & 2 & 1 & 0 & 2 \\
1 & 1 & 2 & 1 & 1 \\
2 & 0 & 1 & 2 & 0 \\
0 & 2 & 1 & 0 & 2 
\end{bmatrix}  $
\dots $R_PC$

$ \forall \ R_pC \ scoreT \ is \ evaluated \ with S=0.6$

$ scoreTT_1=\begin{bmatrix}
1 \\
1 \\
0 \\
1 \\
1 
\end{bmatrix}  $
\\ This means that in permutation 1 the third cell is unstable "jumping" from cluster number 1 in $P_0$ to cluster number 2 in $P_1$.

$Score_m$ is evaluated then for each p\\ 

$ Score_{mp}=\begin{bmatrix}
1 & 0 & 1 & 1 \\
1 & 0 & 1 & 1 \\
0 & 0 & 0 & 1 \\
1 & 0 & 1 & 1 \\
1 & 0 & 1 & 1 
\end{bmatrix}  $
$ Score_m=\begin{bmatrix}
0.6 \\
0.6 \\
0.25 \\
0.6 \\
0.6 
\end{bmatrix}  $

In this example number of cell in $P_0$ is the same as all the other permutation. What happen with real data is that $P_p$ has a \% of cell less then the $P_0$. The cells that are not presents in $P_p$ are then removed from $P_0$ and will be added with a 0 score in the specific position of the matrix ScoreTT. 
\bibliographystyle{abbrv}
\bibliography{main}

\end{document}
This is never printed